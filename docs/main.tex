\documentclass[12pt, twoside, hidelinks, a4paper]{article}


\usepackage[]{geometry}
\geometry{inner=30mm, outer=20mm, top=25mm, bottom=25mm}

\usepackage{mystyle}
\pagestyle{headings}

\usepackage{fancyhdr}
\fancyhf{}
\pagestyle{fancy}
\renewcommand{\headrulewidth}{0pt}
% numery stron: lewa do lewego, prawa do prawego
\fancyfoot[LE,RO]{\thepage}

\fancypagestyle{plain}
{
   \fancyhf{}
\renewcommand{\headrulewidth}{0pt}
% numery stron: lewa do lewego, prawa do prawego
\fancyfoot[LE,RO]{\thepage}
}

\usepackage{pdfpages}

%\renewcommand{\familydefault}{\sfdefault}
\setlength\parindent{1cm}

\usepackage{indentfirst}
\usepackage[affil-it]{authblk}
\usepackage{smartdiagram}
\usepackage{metalogo}

\begin{document}
    \setstretch{1.15}
 	\pagenumbering{arabic}
    %\include{chapters/abstract}

\author{Artur Błaszczyk, Artur Zygadło, Marcin Waszak}
\title{Sprawozdanie projektu 1. z przedmiotu PORR}
\date{2 grudnia 2018}
\affil{Wydział Elektroniki i Technik Informacyjnych, Politechnika Warszawska}


\maketitle

\begin{abstract}
Celem sprawozdania jest przedstawienie przykładowych sposobów implementacji zrównoleglonego algorytmu wyszukiwania liczb pierwszych. W dalszej częsci znajduje się prezentacja rezultatów i wnioski. 
\end{abstract}

\section{Wstęp}
Lorem ipsum...

\printbibliography

\end{document}
